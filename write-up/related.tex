The premier (and only statistically robust) calculation of a hot Jupiter occurrence rate from transit data is that from the primary {\it Kepler} mission \cite{howard2012}.  Because of the exquisite quality of the data from the {\it Kepler} telescope, their detection efficiency for hot Jupiters is near 100\% and they are better able to automatically identify false positives than the HAT collaboration is from our surveys.  Because of this, the {\it Kepler} pipeline needs much less manual vetting of planet candidates and thus the {\it Kepler} team has not needed to worry about the manual step as teh HAT collaboration does.  However, as mentioned before, the HAT collaboration has found more hot Jupiters than any other collaboration, and so if we can get a good handle on our detection efficiency we have the potential to provide the most precise measure of the occurrence rate of these planets, which will help inform planet formation theory.

Unfortunately, since the {\it Kepler} work did not need to worry so much about detection efficiencies and pipeline misclassifications, they did not develop the hard statistical/machine learning tools that would be useful for this work.  However, a recent astrophysics study \cite{ogle2017} that used a very similar time-series data set as the HAT surveys used a random forest classifier to classify 450,000 eclipsing binaries (binary stars that transit each other, similar to how a planet transits a star) from their data.