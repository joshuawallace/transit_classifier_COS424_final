The data used are unpublished and (currently) proprietary.  They were extracted from the HAT database using a script provided by Joel Hartman\footnote{jhartman@astro.princeton.edu}, Research Scienctist at Princeton University.  The database itself currently has no reference.

The data consists of 1024 positive cases and 29,421 negative cases of potential planet candidates that passed the manual vetting process.  The number of positive cases is perhaps too low to train a truly robust model, but it is unfortunately all we have to work with.  The process of discovering new planets is one of constant attrition, and there are unfortunately more cases of false positive signals than of real signals.  There are 92 features that consist of such measured quantities as the fit parameters to the transit dip (transit depth, length, etc.), quantities related to the power spectrum from the matched filtering, measured quantities for the star (brightness in various colors), and signal to noise measures.  All of these quantities are  continuous, not categorical.  There are some missing values: 284 ``inf'' (0.01\% of all values) and 4461 ``nan'' (0.2\% of all values).  The ``inf'' values are mostly concentrated in a single column (feature) of the data, while the ``nan'' values tend to cluster in the same rows (objects).  These data were imputed using [fill in by Will].