Since the discovery of the first extra-solar planet (exoplanet) in the early 90's \cite{pulsarplanet}, nearly 3000 exoplanets have been discovered \cite{exoplanetorg}, with potentially thousands more exoplanets remaining unconfirmed or as yet undiscovered in existent data.  This explosion in exoplanet discovery has been fueled by the funding and construction of many exoplanet detection surveys.  The majority of exoplanets have been found by the space-based {\it Kepler} telescope and its primary and ``K2'' surveys \cite{kepler}, \cite{k2}.  Ground-based surveys, while not as sensitive to smaller planets as {\it Kepler} (due to atmospheric distortions and other problems associated with observing from the ground), are sufficiently sensitive to discover Jupiter-sized planets on short-period orbits.  (Such planets are called ``hot Jupiters'' because they are the same size as Jupiter but are sufficiently close in to the stars they orbit (much closer than Mercury is to our sun) that the star heats them up to several thousands of degrees Celsius.)  Since ground-based surveys are much cheaper to run per area of sky monitored than spaced-based surveys, large-scale ground-based surveys have discovered more hot Jupiters than space-based surveys.

Of the few hundred hot Jupiters that have been found to date, the plurality (${\sim}$100) have been found by Princeton's own Hungarian Automated Telescope (HAT) collaboration, headed by Prof. Gaspar Bakos of the Department of Astrophysical Sciences.  This collaboration runs two surveys, both of which use telescopes that are no bigger than large camera lenses.  The longest-running of the two surveys is HAT-Net, a collection of five telescopes in Arizona and two telescopes in Hawaii \cite{hatnet}.  The other survey is HAT-South, which consists three locations with eight telescopes each: Chile, Namibia, and Australia \cite{hatsouth}.

Planets are far too faint compared to their host stars to be able to detect directly.  Instead, what the HAT arrays do is monitor the brightness of stars as a function of time.  When a planet crosses in front the star it orbits, the amount of light detected from that star is decreased.  When a star is found to have periodic dips in brightness, it is labelled as a potentially planet-hosting star.  Various cuts are made to help ensure that the signal is neither spurious nor caused by another astrophysical source other than a planet.  Most of these cuts are automated, but the final step in the HAT pipeline is a manual, by-eye examination of the signal.  Planet candidates that pass this step then are sent for follow up (and hopefully confirmation!) at bigger telescopes.

Other than this manual step, the vetting of possible planet candidates is completely automatic.  This single manual step is quite labor intensive, and prevents a fully automatic characterization of planet detection efficiency, which is necessary for the calculation of statistics related to the occurrence rate of these planets.  Thus, it would be nice to have an automated approximation to the by-eye work that has occurred.  This is the purpose of this project.