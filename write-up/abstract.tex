A proposal to use machine learning techniques to aid in the binary classification of data from the Hungarian Automated Telescope (HAT) arrays.  The HAT arrays are designed to detect periodic dimmings of stars, potentially caused by orbiting planets around those stars.  There are many signals, both physical and spurious, that can approximate planet signals, so it is important to weed out as many false positives as possible.  After an automated pipeline removes probable false planets, a final manual by-eye selection is made to create the final set of planet cadidates.  We propose to train classifier models to approximate this by-eye selection with the hope of finding a sufficiently well-performing model that can stand as a proxy for or even entirely replace the manual portion of the selection pipeline.